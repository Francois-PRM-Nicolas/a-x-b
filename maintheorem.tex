% -*- mode: latex; eval: (flyspell-mode 1); ispell-local-dictionary: "american"; TeX-master: "preambule"; -*-

\section{proof of the main theorem}


   $S(0, \ttx) = 0$, $S(1, \ttx) = 1$, $S(2, \ttx) = 1 + \ttx$, $S(3, \ttx) = 1 + \ttx + \ttx^2$.

   $$
   S(n + 1, \ttx) = \ttx S(n, \ttx) + 1
   $$

   $$
   S(mn, \ttx) = S(m, \ttx^n) S(n, \ttx)
   $$

   

   Theorem~\ref{thm:p-odd-pcarre} is merely a lemma.

 \begin{theorem} \label{thm:p-odd-pcarre}
   Let $p \in \bN$ and let $q \in \bZ$.
   If $p$ is odd or if $q$ is even then
   \begin{equation} \label{eq:Spa-mod-psquare}
     S(p, 1 + pq) \equiv p \pmod {p^2}\, .
   \end{equation}
 \end{theorem}


 \begin{proof}
   Put 
   $$
   f(\ttx) = \sum_{k = 3}^p \binom{p}{k} \ttx^{k - 3} 
   $$
   and
   $$
   u =  \frac{(p - 1)q}{2}  + q^2 f(p q) \,.
   $$
   Note that $f(\ttx) \in \bZ[\ttx]$ and $u \in \bZ$.
   Equation~\eqref{eq:Spa-mod-psquare} holds trivially true if $p \in \{ 0, 1 \}$,
   so we may assume $p \ge 2$.
     Then, the binomial theorem yields
   $$
   {(1 + \ttx)}^p
   =
   1 + p \ttx + \frac{p(p - 1)}{2} \ttx^2 + \ttx^3 f(\ttx) \,, 
   $$
   whence
   $$
   S(p, 1 + \ttx)
   =
   \frac{{(1 + \ttx)}^p - 1}{\ttx}
   =
   p + \frac{p(p - 1)}{2} \ttx + \ttx^2 f(\ttx) \,.
   $$
   It follows $S(p, 1 + p q) =  p + u p^2$,
   and thus Equation~\eqref{eq:Spa-mod-psquare} holds true.
 \end{proof}

 
 \begin{theorem} \label{thm:p-odd-pk}
   Let $k$, $n$, $p \in \bN$ and let $q \in \bZ$.
   If $p$ is odd or if $q$ is even then
   \begin{equation} \label{eq:Spa-mod-pk}
     S(n p^k, 1 + p q) \equiv n p^k \pmod {p^{k + 1} } \,.
   \end{equation}
 \end{theorem}


 \begin{proof}
   We proceed by induction on~$k$.
   Put $a = 1 + p q$,  $m = n p^k$, and $q' = q S(m, a)$.
   Every power of $a$ is congruent to $1$ modulo $p$
   and
   $S(n, a)$ can be written as the sum of $n$ powers of~$a$.
   It follows $S(n, a) \equiv n \pmod {p}$, whence
   Equation~\eqref{eq:Spa-mod-pk} holds true for $k = 0$.

   Since  
   $$
   a^m - 1 = (a - 1) S(m, a) 
   $$
   we have 
   $$
   a^m  = 1 + p q'  \,, 
   $$
   and thus
   
   Now, assume that Equation~\eqref{eq:Spa-mod-pk} holds true for some $k \in bN$.
   Then, $p^{k + 1}$ divides $S(m, a) - m$.
   Since
   $$
   S(m p, a) = S(m, a) S(p, a^m) \, , 
   $$
   $$
   S(m p, a) - m p = (S(m, a) - m) p + S(m, a)(S(p, a^m) - p) 
   $$

   $$
   S(p m, a) - p m = S(p, a^m)(S(m, a) - m) + (S(p, a^m) - p) m
   $$
   
   $$
   \nu_p(S(m p, a) - m p) \ge  \min \left\{ 1 + \nu_p(S(m, a) - m), 2 + \nu_p(m)  \right\}
   $$
 \end{proof}
   


 Let $p$ be a prime.
 For every $s$, $n \in \bZ$, 
 $\nu_p(s - n) > \nu_p(n)$ implies $\nu_p(s) = \nu_p(n)$.
 The converse holds true if $p = 2$ and $n \ne 0$.

 \begin{theorem} \label{thm:val-adic}
   Let $n \in \bNast$ and let $q \in \bZ$.
   \begin{enumerate}
   \item For each odd prime $p$, the $p$-adic valuation of $S(n, 1 + p q) - n$ is larger than that of~$n$.
   \item The $2$-adic valuation of $S(n, 1 + 4 q)$ is equal to that of~$n$.
   \end{enumerate} 
 \end{theorem}

 $$
 S(p n, a) = S(p, a^n) S(n, a) 
 $$

 $$
 S(np, a)  - np =  S(n, a) (S(p, a^n) - p) + (S(n, a) - n) p
 $$

 $$
 \nu_p(S(n, a) - n) > \nu_p(n) 
 $$
 $$
 \nu_p((S(n, a) - n) p) > \nu_p(np) 
 $$
 $$
\nu_p( S(n, a) (S(p, a^n) - p) ) = \nu_p(S(n, a)) + \nu(S(p, a^n) - p) \ge \nu_p(np) = 
 $$
 $$
 \nu_p(S(pn, a) - pn) 
 $$

 % The $p$-adic valuation of $\nu_p(S(pn, a) - pn) is larger than $S(n, a) - n$
 % $p^{k + 1}$ divides $S(pn, a) - pn$
 
 \begin{proof}
   We prove both parts at once.
   Put $T(k, \ttx)  = S(k, \ttx) - k$ for each $k \in \bNast$.
   Let $p$ be a prime factor of $a - 1$.
   Assume that $p$ is odd or that $4$ divides $a - 1$.
   Our task is to prove
   \begin{equation}  \label{eq:val-adic:nup}
   \nu_p(T(n, a)) > \nu_p(n) \, . 
   \end{equation}
   We proceed by induction on the total number of prime factors of~$n$.

   First, assume that $n$ has no prime factors.
   Then, we have $n = 1$, and consequently, $T(n, a)  = 1 - 1 = 0$.
   It follows $\nu_p(n) = 0 < \infty = \nu_p(T(n, a))$,
   and thus Equation~\eqref{eq:val-adic:nup} holds true.

   Second, assume $n = p = 2$.
   Then, we have
   $\nu_p(n) = \nu_2(2) = 1$
   and 
   $T(n, a) = S(2, a) - 2 = a + 1 - 2 = a - 1$.
   Moreover, $4$ divides $a - 1$ because $p$ is not odd,
   whence $\nu_p(T(n, a)) = \nu_2(a - 1) > 1$.
   Therefore, Equation~\eqref{eq:val-adic:nup} holds true.

   Third, assume $n = p \ne 2$.
   Then,
   we have $\nu_p(n) = \nu_p(p) = 1$
   and
   Theorem~\ref{thm:p-odd-pcarre} ensures that Equation~\eqref{eq:Spa-mod-psquare} holds true.
   Besides, the latter is equivalent to $\nu_p(T(n, a)) \ge 2$.
   Therefore, Equation~\eqref{eq:val-adic:nup} holds true.

   Fourth, every power of $a$ is congruent to $1$ modulo $p$
   and
   $S(n, a)$ can be written as the sum of $n$ powers of $a$.
   It follows $S(n, a) \equiv n \pmod {p}$, or equivalently, $\nu_p(T(n, a)) \ge 1$.
   Therefore, Equation~\eqref{eq:val-adic:nup} holds true whenever $p$ does not divide~$n$.
 
   At this point of our discussion,
   we have proven that Equation~\eqref{eq:val-adic:nup} holds true if $n = 1$ or if $n$ is prime,
   so it only remains to deal with the case where $n$ is composite.
   Assume that there exist $m$, $m' \in \bNdeux$ such that $n = mm'$.
   Put $\mu = \nu_p(m)$ and $\mu' = \nu_p(m')$: $\mu\mu' = \nu_p(n)$.
   $$
   \nu(T(m, a)) > \mu
   $$
   $$
   \nu(S(m, a) = \mu
   $$
   $$
   \nu(T(m', a^m)) > \mu'
   $$
   $$
   \nu(T(m, a) m') > \mu + \mu' 
   $$
   $$
   T(mm', a) = T(m, a) m' +  S(m, a) T(m', a^m) 
   $$
 \end{proof}

   $$
   S(nn', 1 + p \ttx) 
   =
   S(n, 1 + p \ttx)  S(n', {(1 + p \ttx)}^n)
   $$

 