% -*- mode: latex; eval: (flyspell-mode 1); ispell-local-dictionary: "american"; TeX-master: "preambule"; -*-

\section{proof of the main theorem}


   $S(0, \ttx) = 0$, $S(1, \ttx) = 1$, $S(2, \ttx) = 1 + \ttx$, $S(3, \ttx) = 1 + \ttx + \ttx^2$.

   $$
   S(n + 1, \ttx) = \ttx S(n, \ttx) + 1
   $$

   $$
   S(mn, \ttx) = S(m, \ttx^n) S(n, \ttx)
   $$

\cite{HullDobell62}   

Theorem~\ref{thm:p-odd-pcarre} is merely a lemma.

\begin{theorem}
  For every $m$, $n \in \bZ$ such that $\rho'(m)$ divides $n$,
  $2 \rho(m)$ divides $n$ or $2 \rho(m)$ divides $m + n$.
\end{theorem}

\begin{proof}
  % $\rho(m)$ divides $\rho(m) + $
  % Assume $\nu(m) < \nu(n)$.
  % Then $2 \rho(m)$ divides~$n$.
  % If $\nu(m) = \nu(n)$ then  $\nu(m + n) > \nu(m)$ and $2 \rho(m)$ divides~$n$.
  % If $\nu(m) < \nu(n)$ 
  %Assume that $\rho'(m)$ divides~$n$.
  % If $n$ is even then $2 \rho(m)$ divides~$n$.
  % If $n$ is odd then $2 \rho(m)$ divides $m + n$.
  % Assume that $\rho(m)$ is odd.
  % If $n$ is even then 
  % If $n$ is odd then
 % Assume that $\rho'(m)$ divides~$n$.
%  Then, $\rho(m)$ divides both $n$ and $m + n$.
%  Assume that $m$ and $\rho(m)$ are odd.
%  If $n$ is even then $2 \rho(m)$ divides~$n$.
%  If $n$ is odd  then $2 \rho(m)$ divides $m + n$.
  
  % If $\rho(m)$ is even  divides $m$ 
  % Assume that $\rho'(m) \ne 2 \rho(m)$ and that $\rho'(m)$ divides~$n$.
%  Assume that $m$ is odd.
  If $m$ is odd and if $n$ is even then $2 \rho(m)$ divides~$n$.
  If both $m$ and $n$ are odd then $2 \rho(m)$ divides $m + n$.
  If $4$ divides $n$ then $2 \rho(m)$ divides~$n$.
  %If $4$ divides $m + n$ then $2 \rho(n)$ divides $m + n$.
  Finally, assume that $m$ is even and that $4$ does not divide~$n$.
  Then, $4$ does not divides $m$, $4$ divides $m + n$, and thus $2 \rho(m)$ divides $m + n$.
\end{proof}

\begin{theorem}
  For each $r \in \bN$, there exists $f_r(\ttx) \in \bZ[\ttx]$ such that 
  $$
  {(1 + \ttx)}^r = 1 + r \ttx + \frac{r(r - 1)}{2} \ttx^2 + \ttx^3 f_r(\ttx) \, .
  $$
\end{theorem}

\begin{proof}
 
 
 
 \end{proof} 

 \begin{theorem}
   Let $a$, $r \in \bZ$ be such that $a$ and is invertible modulo~$r$.
   For each $k \in \bN$, $a$ is invertible modulo~$r^k$.
 \end{theorem}
 
 \begin{proof}
   By assumption, there exists $b \in \bZ$ such that $r$ divides $1 - ab$.
      $(1 - ab)(1 - ab')  = 1 - a(b + b' - b b')$
 \end{proof}
 
 \begin{theorem} \label{thm:p-odd-pcarre}
   For each $r \in \bN$,
   $S(r, 1 + 2r)$ is congruent to $r$ modulo $r^2$.
 \end{theorem}

 \begin{theorem} \label{thm:p-odd-pcarre}
   For every $n \in \bN$ and every $b \in \bZ$,
   $S(n, 1 + 2b)$ is congruent to $n$ modulo $\gcd(n, b) b$.
 \end{theorem}

\begin{proof}
   Put
   $$
   f(\ttx) = \sum_{k = 3}^{np} \binom{np}{k} \ttx^{k - 3} 
   $$
   and $k = n(np - 1) + 4 q f(2 p q)$.
   By construction, $f(\ttx)$ lies in $\bZ[\ttx]$, $k$ is an integer, and the binomial theorem ensures 
   $$
   {(1 + \ttx)}^{np} = 1 + np \ttx + \frac{np(np - 1)}{2} \ttx^2 + \ttx^3 f(\ttx) \,.
   $$
   It follows
   $$
   S(np, 1 + \ttx) = \frac{{(1 + \ttx)}^{np} - 1}{\ttx} = np + \frac{np(np - 1)}{2} \ttx + \ttx^2 f(\ttx) \,,
   $$
   and consequently,
   $$
   S(np, 1 + 2pq) = np + \frac{np(np - 1)}{2}(2pq) + {(2pq)}^2f(2pq) = np + kp^2q \,. 
   $$
 \end{proof}

   $$
   \frac{S(np, 1 + 2pq)}{p} = n + kpq
   $$

   $$
   S(np, 1 + 2pq) = (n + kpq) p = S(n, 1 + 2pq) S(p, 1 + 2pq) 
   $$
   
 \begin{theorem} \label{thm:p-odd-pcarre}
   For every $p \in \bN$ and every $q \in \bZ$,
   $S(p, 1 + 2pq)$ is congruent to $p$ modulo $p^2q$.
 \end{theorem}

\begin{proof}
   Put
   $$
   f(\ttx) = \sum_{k = 3}^p \binom{p}{k} \ttx^{k - 3} 
   $$
   and $n = p - 1 + 4 q f(2 p q)$.
   By construction, $f(\ttx)$ lies in $\bZ[\ttx]$,
   $n$ is an integer, and
   the binomial theorem ensures 
   $$
   {(1 + \ttx)}^p = 1 + p \ttx + \frac{p(p - 1)}{2} \ttx^2 + \ttx^3 f(\ttx)  \, .
   $$
   It follows
   $$
   S(p, 1 + \ttx) = \frac{{(1 + \ttx)}^p - 1}{\ttx} = p + \frac{p(p - 1)}{2} \ttx + \ttx^2 f(\ttx) \,,
   $$
   and consequently,
   $$
   S(p, 1 + 2 pq) = p + \frac{p(p - 1)}{2} (2 p q) + {(2 p q )}^2 f(2 p q) = p + n p^2 q \,. 
   $$
 \end{proof}
 $$
 S(pq, \ttx) =  S(p, \ttx^q) S(q, \ttx)   
 $$
 $$
 (1 +  \ttx)^q   = 1 + \ttx S(q, 1 + \ttx)
 $$
 $$
 S(pq, 1 + \ttx)
 =  S(p, {(1 + \ttx)}^q) S(q, 1 + \ttx)
 =  S(p, 1 + \ttx S(q, 1 + \ttx)) S(q, 1 + \ttx)
 $$
 \begin{theorem}
 \end{theorem}
 
 \begin{theorem}
   Let $m$, $n \in \bZ$ and let $X$, $Y \subseteq \bZ$.
   %Let $f \colon X \to \bZ$ such that $f(x)$ is invertible modulo $n$ for every $x \in X$.
   If $X$ is a complete residue system modulo $m$ and
   if $Y$ is a complete residue system modulo $n$
   then
   $x + m Y$ is a complete residue system modulo $mn$.
 \end{theorem}
 \begin{theorem}
   Let $m$, $n \in \bZ$ and let $X$, $Y \subseteq \bZ$.
   Let $f \colon X \to \bZ$ such that for each $x \in X$, $f(x)$ is invertible modulo~$n$.
   If $X$ is a complete residue system modulo $m$ and
   if $Y$ is a complete residue system modulo $n$
   then
   $$
   \left\{ x +  f(x) y : (x, y) \in X \times Y \right\} 
   $$
 \end{theorem}

 $$
  \left\{ x + f(x) m y  \right\} 
 $$

 \begin{proof}
%   Without loss of generality, we may assume that both $m$ and $n$ are non-negative. 
   If $m = 0$ then $X + m Y = X = \bZ$ is the unique complete residue system modulo $0 = mn$.
   Therefore, we may assume $m \ne 0$.
   If $n \ne 0$ then the cardinality of $X + m Y$ is not greater than $|mn|$.
   Therefore, it suffices to prove that for each $t \in \bZ$,
   there exists $z \in X + m Y$ such that $t \equiv z \pmod{mn}$.
   Let $t \in \bZ$.
   Since $X$ is a complete residue system modulo $m$,
   there exists $x \in X$ such that $t \equiv x \pmod{m}$.
%   $f(x) - k n = m$
   Since $Y$ is a complete residue system modulo $n$,
   there exists $y \in Y$ such that $(t - x) m^{-1}  \equiv y \pmod{n}$.
   Put $z = x + m y$.
   By construction, $z$ satisfies $z \in X + m Y$ and $t \equiv z \pmod{mn}$.
\end{proof} 
 

 \begin{theorem}
   Let $a \in \bZ$ and let 
   For every $k \in \bN$ and every $r$, $y \in \bZ$, 
   there exists $x \in \bZ$ such that $y = S(x, 1 + 2r) \pmod {r^k}$.  
 \end{theorem}

 \begin{proof}
   $r = \rho(m)$, $r' = \rho'(m)$
   $$
   a_k = a^{r^k} 
   $$

   $$
   a_{k + 1} = a_k^r 
   $$

   Every power of $a$ is congruent to $1$ modulo~$r'$.

   $a \equiv 1 \pmod{r'}$
   
   $a_k \equiv 1 \pmod{r'}$ implies $a_{k + 1} \equiv 1 \pmod{r'}$.
    
   $$
   S(r, a_k^n) S(n, a_k) = S(r n, a_k) = S(r, a_k) S(n, a_{k + 1})
   $$

   $$
   \frac{S(r, a_k^n)}{r} S(n, a_k) = \frac{S(r, a_k)}{r} S(n, a_{k + 1})
   $$

   $$
   a_{k + 1} - 1 = a_k^r - 1 = (a_k - 1) S(r, a_k) 
   $$

   $r^k$ divides $a_k  - 1$

  
 \end{proof}
 
 
   

 
 \begin{theorem} \label{thm:p-odd-pk}
   Let $k$, $n$, $p \in \bN$ and let $q \in \bZ$.
   If $p$ is odd or if $q$ is even then
   \begin{equation} \label{eq:Spa-mod-pk}
     S(n p^k, 1 + p q) \equiv n p^k \pmod {p^{k + 1} } \,.
   \end{equation}
 \end{theorem}


 \begin{proof}
   We proceed by induction on~$k$.
   Put $a = 1 + p q$,  $m = n p^k$, and $q' = q S(m, a)$.
   Every power of $a$ is congruent to $1$ modulo $p$
   and
   $S(n, a)$ can be written as the sum of $n$ powers of~$a$.
   It follows $S(n, a) \equiv n \pmod {p}$, whence
   Equation~\eqref{eq:Spa-mod-pk} holds true for $k = 0$.

   Since  
   $$
   a^m - 1 = (a - 1) S(m, a) 
   $$
   we have 
   $$
   a^m  = 1 + p q'  \,, 
   $$
   and thus
   
   Now, assume that Equation~\eqref{eq:Spa-mod-pk} holds true for some $k \in bN$.
   Then, $p^{k + 1}$ divides $S(m, a) - m$.
   Since
   $$
   S(m p, a) = S(m, a) S(p, a^m) \, , 
   $$
   $$
   S(m p, a) - m p = (S(m, a) - m) p + S(m, a)(S(p, a^m) - p) 
   $$

   $$
   S(p m, a) - p m = S(p, a^m)(S(m, a) - m) + (S(p, a^m) - p) m
   $$
   
   $$
   \nu_p(S(m p, a) - m p) \ge  \min \left\{ 1 + \nu_p(S(m, a) - m), 2 + \nu_p(m)  \right\}
   $$
 \end{proof}
   


 Let $p$ be a prime.
 For every $s$, $n \in \bZ$, 
 $\nu_p(s - n) > \nu_p(n)$ implies $\nu_p(s) = \nu_p(n)$.
 The converse holds true if $p = 2$ and $n \ne 0$.

 \begin{theorem} \label{thm:val-adic}
   Let $n \in \bNast$ and let $q \in \bZ$.
   \begin{enumerate}
   \item For each odd prime $p$, the $p$-adic valuation of $S(n, 1 + p q) - n$ is larger than that of~$n$.
   \item The $2$-adic valuation of $S(n, 1 + 4 q)$ is equal to that of~$n$.
   \end{enumerate} 
 \end{theorem}

 $$
 S(p n, a) = S(p, a^n) S(n, a) 
 $$

 $$
 S(np, a)  - np =  S(n, a) (S(p, a^n) - p) + (S(n, a) - n) p
 $$

 $$
 \nu_p(S(n, a) - n) > \nu_p(n) 
 $$
 $$
 \nu_p((S(n, a) - n) p) > \nu_p(np) 
 $$
 $$
\nu_p( S(n, a) (S(p, a^n) - p) ) = \nu_p(S(n, a)) + \nu(S(p, a^n) - p) \ge \nu_p(np) = 
 $$
 $$
 \nu_p(S(pn, a) - pn) 
 $$

 % The $p$-adic valuation of $\nu_p(S(pn, a) - pn) is larger than $S(n, a) - n$
 % $p^{k + 1}$ divides $S(pn, a) - pn$
 
 \begin{proof}
   We prove both parts at once.
   Put $T(k, \ttx)  = S(k, \ttx) - k$ for each $k \in \bNast$.
   Let $p$ be a prime factor of $a - 1$.
   Assume that $p$ is odd or that $4$ divides $a - 1$.
   Our task is to prove
   \begin{equation}  \label{eq:val-adic:nup}
   \nu_p(T(n, a)) > \nu_p(n) \, . 
   \end{equation}
   We proceed by induction on the total number of prime factors of~$n$.

   First, assume that $n$ has no prime factors.
   Then, we have $n = 1$, and consequently, $T(n, a)  = 1 - 1 = 0$.
   It follows $\nu_p(n) = 0 < \infty = \nu_p(T(n, a))$,
   and thus Equation~\eqref{eq:val-adic:nup} holds true.

   Second, assume $n = p = 2$.
   Then, we have
   $\nu_p(n) = \nu_2(2) = 1$
   and 
   $T(n, a) = S(2, a) - 2 = a + 1 - 2 = a - 1$.
   Moreover, $4$ divides $a - 1$ because $p$ is not odd,
   whence $\nu_p(T(n, a)) = \nu_2(a - 1) > 1$.
   Therefore, Equation~\eqref{eq:val-adic:nup} holds true.

   Third, assume $n = p \ne 2$.
   Then,
   we have $\nu_p(n) = \nu_p(p) = 1$
   and
   Theorem~\ref{thm:p-odd-pcarre} ensures that Equation~\eqref{eq:Spa-mod-psquare} holds true.
   Besides, the latter is equivalent to $\nu_p(T(n, a)) \ge 2$.
   Therefore, Equation~\eqref{eq:val-adic:nup} holds true.

   Fourth, every power of $a$ is congruent to $1$ modulo $p$
   and
   $S(n, a)$ can be written as the sum of $n$ powers of $a$.
   It follows $S(n, a) \equiv n \pmod {p}$, or equivalently, $\nu_p(T(n, a)) \ge 1$.
   Therefore, Equation~\eqref{eq:val-adic:nup} holds true whenever $p$ does not divide~$n$.
 
   At this point of our discussion,
   we have proven that Equation~\eqref{eq:val-adic:nup} holds true if $n = 1$ or if $n$ is prime,
   so it only remains to deal with the case where $n$ is composite.
   Assume that there exist $m$, $m' \in \bNdeux$ such that $n = mm'$.
   Put $\mu = \nu_p(m)$ and $\mu' = \nu_p(m')$: $\mu\mu' = \nu_p(n)$.
   $$
   \nu(T(m, a)) > \mu
   $$
   $$
   \nu(S(m, a) = \mu
   $$
   $$
   \nu(T(m', a^m)) > \mu'
   $$
   $$
   \nu(T(m, a) m') > \mu + \mu' 
   $$
   $$
   T(mm', a) = T(m, a) m' +  S(m, a) T(m', a^m) 
   $$
 \end{proof}

   $$
   S(nn', 1 + p \ttx) 
   =
   S(n, 1 + p \ttx)  S(n', {(1 + p \ttx)}^n)
   $$
