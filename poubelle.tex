
%%% Local Variables:
%%% mode: latex
%%% TeX-master: "preambule"
%%% End:

  % Remark that equality 
  % \begin{equation} \label{eq:interval-slide}
  %   \sego{n}{n + m} \setminus \{ n \}
  %   %= \sego{n + 1}{n + m}
  %   = \sego{n + 1}{n + m + 1} \setminus \{ n + m \}
  %   %%\,.
  % \end{equation}
  % holds true.
  % Since $g$ induces a bijection from $\sego{n}{n + m}$ onto $X$,
  % $g$ maps the left-hand side of Equation~\eqref{eq:interval-slide} onto $X \setminus \{ g(n) \}$;
  % since  $g$ induces a bijection from $\sego{n + 1}{n + m + 1}$ onto $X$,
  % $g$ maps the right-hand side of Equation~\eqref{eq:interval-slide} onto $X \setminus \{ g(n + m) \}$.
  % It follows
  % $$X \setminus \{ g(n) \} = X \setminus \{ g(n + m) \}\,,$$
  % or equivalently, $g(n) = g(n + m)$.
  % Therefore, $m$ is a period of $g$,
  % Since the restriction of $g$ to $\bNast$ is surjective, $P$ is not equal to $\{ 0 \}$.
  % Since $g(n + 1) = f(g(n))$ for every $n \in \bN$, $P$ is the set of all periods of~$g$.
  % Let $K$ denote the set of those $n \in \bNast$ such that $g$ is injective on $\sego{0}{n}$.
  % Let $k$ denote the greatest element of~$K$.
  % By construction, $g$ is injective on $\sego{0}{k}$ and $g(k)$ is an element of $g(\sego{0}{k})$.

  % If $g(k) \ne g(0)$ then $f(g(\sego{0}{k})) = g(\sego{1}{k + 1}) = g(\sego{1}{k})$.
  $\ref{ass:def-cyclic-enumeration:surj-f} \implies  \ref{ass:def-cyclic-enumeration:surj-periodic}$.
   Put
   $$
   N = \left\{ n \in \bN : g(n) \in g(\sego{0}{n}) \right\} \, .
   $$
   Since $g$ is not injective, $N$ is non-empty;
   let $s$ denote the least element of~$N$.
   By construction,
   there exists $r \in \sego{0}{s}$ such that $g(r) = g(s)$ and 
   $g$ is injective on $\sego{0}{s}$.
   Assume that assertion~\ref{ass:def-cyclic-enumeration:surj-f} holds true.
   Since $g(r)$ is an element of $g(\sego{r}{s})$ and since $g(\sego{r}{s})$ is closed under $f$,
   a straightforward induction on $n$ shows that $g(n)$ belongs to $g(\sego{r}{s})$ for every $n \in \bZ_{\ge r}$,
   whence $g(\bZ_{\ge r}) = g(\sego{r}{s})$.
   Besides, there exists $t \in \bNast$ such that $g(t) = g(0)$.
    $t$ is larger than $s$, and thus $g(0)$ belongs to $g(\seg{r}{s})$, 
   $$X = g(\bNast) = g(\sego{1}{r}) \cup g(\bZ_{\ge r}) = g(\sego{1}{r}) \cup g(\sego{r}{s}) $$
  
   Moreover, $g$ is surjective by assumption.
   % Therefore, the restriction of $g$ to $\sego{0}{s}$ is also surjective.
   Therefore, we have $m = s$ and $g$ induces a bijection from $\sego{0}{m}$ onto~$X$.
   Besides, there exists $t \in \bNast$ such that $g(t) = g(0)$.
   Hence, there exists $u \in \sego{0}{m}$ such that $g(u) = g(t - 1)$.
   Since
   $$g(u + 1) = f(g(t - 1)) = g(t) = g(0) \, ,$$
   the injectivity of $g$ on $\sego{0}{m}$ compels $u + 1 = m$,
   and consequently, $g(m) = g(0)$.
   Now, a straightforward induction on $m$ shows that $g(n + m) = g(n)$ for every $n \in \bN$,
   and thus assertion~\ref{ass:def-cyclic-enumeration:surj-periodic} holds true.
   Put
   $$
   N = \left\{ n \in \bN : g(n) \in g(\sego{0}{n}) \right\} \, .
   $$
   Since $g$ is not injective, $N$ is non-empty;
   let $s$ denote the least element of~$N$.  
   By construction,
   $g(s)$ belongs to $g(\sego{0}{s})$ and
   $g$ is injective on $\sego{0}{s}$.
   Assume that assertion~\ref{ass:def-cyclic-enumeration:surj-f} holds true.
   Since $g(0)$ is an element of $g(\sego{0}{s})$ and since $g(\sego{0}{s})$ is closed under $f$,
   a straightforward induction on $n$ show that $g(n)$ belongs to $g(\sego{0}{s})$ for every $n \in \bN$.
   Moreover, $g$ is surjective by assumption.
   % Therefore, the restriction of $g$ to $\sego{0}{s}$ is also surjective.
   Therefore, we have $m = s$ and $g$ induces a bijection from $\sego{0}{m}$ onto~$X$.
   Besides, there exists $t \in \bNast$ such that $g(t) = g(0)$,
   and consequently,
   there exists $u \in \sego{0}{m}$ such that $g(u) = g(t - 1)$.
   Since
   $$g(u + 1) = f(g(u)) = f(g(t - 1)) = g(t) = g(0) \, ,$$
   the injectivity of $g$ on $\sego{0}{m}$ compels $u + 1 = m$,
   and consequently, $g(m) = g(0)$.
   Now, a straightforward induction on $m$ shows that $g(n + m) = g(n)$ for every $n \in \bN$,
   and thus assertion~\ref{ass:def-cyclic-enumeration:surj-periodic} holds true.
   % Assume that assertion~\ref{ass:def-cyclic-enumeration:surj-f} holds true.
   % Since $g(1)$ is an element of $g(\seg{1}{s})$ and since $g(\seg{1}{s})$ is closed under $f$,
   % $g(\seg{1}{s})$ contains $g(\bNast) = X$ as a subset, and thus $s$ is not less than~$m$.
   % Besides, the opposite inequality holds true because $g$ is injective on $\sego{0}{s}$.
   % Hence, we have $m = s$.
   % It follows that 
   % $g$ is injective on $\sego{0}{m}$ and
   % the restriction of $g$ to $\seg{1}{m}$ is surjective
   % It follows $g(m) = g(0)$, and consequently,
   % a straightforward induction on $m$ shows that $g(n + m) = g(n)$ for every $n \in \bN$.
   % Hence, assertion~\ref{ass:def-cyclic-enumeration:surj-periodic} holds true.
 
  a straightforward induction on $n$ shows $g(p - n) \in Y$ for every $n \in \seg{0}{p}$.
  $The set $g(\sego{p}{q})$ contains $g(p)$ as an element and is closed under $f$,
  therefore
  $g(\sego{p}{q})$ contains $g(\bZ_{\ge p})$ as a subset.
  
   For each $p \in \bN$, we have $\bZ_{\ge p + 1} = \bZ_{\ge p} + 1$,   
   whence $F_{p + 1} = f(F_p)$.
   In particular, $f$ maps $F_0$ onto $F_1$.
   Since the latter set is equal to $X$ by assumption, $f$ is surjective.
   Therefore, a straightforward, induction on $p$ shows that
   $F_p = X$ for every $p \in \bN$.
   Let $n \in \bN$.
  Since the cardinality of $\seg{n}{n + m}$ is equal to $m + 1$,
  $g$ is not injective on that set,
  and thus there exist $p$, $q \in \seg{n}{n + m}$ such that $p < q$ and $g(p) = g(q)$.
  By construction, $g(p)$ is an element of $g(\sego{p}{q})$ and $g(\sego{p}{q})$ is closed under~$f$.
  Besides, the closure of $\{ g(p) \}$ under $f$ is equal to 
  It follows
  $$X = g(\bZ_{\ge p}) \subseteq g(\sego{p}{q}) \subseteq g(\sego{n}{n + m}) \, ,$$
  and thus  the restriction $g$ to $\sego{n}{n + m}$ is surjective.
  Put
  $$
  P = \left\{ p \in \bNast : g(p) = g(0) \right\} \, .
  $$
  Assume that assertion~\ref{ass:def-cyclic-enumeration:surj-f} holds true.
  A straightforward induction on $n$ shows that
  equality $g(n + p) = g(n)$ holds true for every $n \in \bN$ and every $p \in P$.
  Therefore $P$ is the set of all positive periods of~$g$.
  Our task is to prove $m \in P$.
  Let $p \in P$ be such that $p > m$.
  Since $g$ is not injective on $\seg{1}{p}$,
  there exist $q$, $q' \in \seg{1}{p}$ such that
  $q < q'$ and $g(q) = g(q')$.
 
  
  By assumption, $P$ is non-empty.
  Moreover, $P$ is closed under addition, whence $P$ is infinite. 
    
  Therefore, 
  
  Since $g$ is not injective, $N$ is non-empty.
  Let $s$ denote the least element of $N$.  %and let $r \in \sego{0}{s}$ be such that $g(r) = g(s)$.
  By construction,
%  $s$ is not greater than $m$,
  $g(s)$ is an element of $g(\sego{0}{s})$ and
  $g$ is injective on $\sego{0}{s}$.
  %In particular, $s$ is not greater than~$m$.
  % 
  % Since $g(0)$ is an element of $g(\sego{0}{s})$ and since $g(\sego{0}{s})$ is closed under $f$ by construction,
  % $g(\sego{0}{s})$ contains $g(\bN) = X$ as a subset.
  % It follows that $g$
  Since $g(1)$ is an element of $g(\seg{1}{s})$ and since $g(\seg{1}{s})$ is closed under $f$,
  $g(\seg{1}{s})$ contains $g(\bNast) = X$ as a subset, and thus $s$ is not less than~$m$.
  Besides, the opposite inequality holds true because $g$ is injective on $\sego{0}{s}$.
  Therefore,
  $s$ is equal to $m$,
  $g$ is injective on $\sego{0}{m}$, and
  $g$ maps $\seg{1}{m}$ onto~$X$.
  %$g(0)$ is an element of $g(\seg{1}{m})$.
  It follows $g(m) = g(0)$,
  and consequently,
  a straightforward induction on $n$ shows that equality $g(n + m) = g(n)$ holds true for every $n \in \bN$.
$$
\sum_{k} S(n_k, a) = \sum_k n_k + \sum_k S(n_k, a) - n_k  
$$

$$
 S(n_k, a) - n_k = (a - 1) \sum_{j = 0}^{n_k - 1}  S(j, a) 
 $$

 $$
\sum_{k} S(n_k, a) =  \sum_k n_k + (a - 1) \sum_k \sum_{j = 0}^{n_k - 1} S(j, a) 
 $$

 $$
 \sum_k \frac{n_k(n_k - 1)}{2} 
 $$
   
   % Let $a \in \bZ$ be fixed.
   % Put 
   % $$
   % D_n = n + \binom{n}{2} (a - 1) - S(n, a) 
   % %D_n = n + \frac{n(n - 1)}{2} (a - 1) - S(n, a) 
   % $$
   % for every $n \in \bN$.
   % Our task is to prove that
   % ${(a - 1)}^2$ divides $D_n$ for every $n \in \bN$.
   % We proceed by induction on~$n$.
   % Since $D_0 = S(0, a) = 0$,
   % the basis of our induction holds true.
   % It remains to check the inductive step.
   % Straightforward computations yields
   % \begin{align*}
   %   D_{n + 1} - a D_n - 1 
   %   & = n + 1 + \binom{n + 1}{2} (a - 1) - a \left(  n + \binom{n}{2} (a - 1) \right) - 1 \\
   %   & = - \frac{n(n - 1)}{2}  {(a - 1)}^2 \,,
   % \end{align*}
   % \begin{align*}
   %   D_{n + 1} - a D_n
   %   & = D_{n + 1} - a D_{n} + \left(  S(n + 1, a) - a S(n, a) - 1 \right)   \\
   %   & = \left( D_{n + 1} + S(n + 1, a) \right)  - a \left(D_n + S(n, a) \right) - 1 \\
   %   & = n + 1 + \frac{(n + 1) n}{2} (a - 1) - a \left(  n + \frac{n(n - 1)}{2} (a - 1) \right) - 1 \\
   %   & = - \frac{n(n - 1)}{2}  {(a - 1)}^2 \,,
   % \end{align*}
   % \begin{align*}
   %   D(n + 1) - a D(n)
   %   & = T(n + 1) - a T(n) -  \left( S(n + 1, a) - a S(n, a) \right)   \\
   %   & = T(n + 1) - a T(n) - 1 \\
   %   & = - \frac{n(n - 1)}{2}  {(a - 1)}^2 \,,
   % \end{align*}
   % whence ${(a - 1)}^2$ divides $D_{n + 1} - a D_n$.
   % Therefore, if ${(a - 1)}^2$ divides $D_n$ then ${(a - 1)}^2$ divides $D_{n + 1}$.
 \end{proof}
 %\begin{proof}                  %
   % Let $B\colon \bZ \to \bZ$ be the univariate polynomial function defined by:
   % $$
   % B(x) = \frac{x (x - 1)}{2}
   % $$
   % for every $x \in \bZ$.
 %   Let $T\colon \bZ \times \bZ \to \bZ$ be defined by:
 %   $$
 %   T(x, y) = x + \frac{x(x - 1)}{2} (y - 1) 
 %   $$
 %   for every $x$, $y \in \bZ$.
 %   Let $a \in \bZ$ be fixed.
 %   Our task is to prove that
 %   ${(a - 1)}^2$ divides $S(n, a) - T(n, a)$ for every $n \in \bN$.
 %   We proceed by induction on~$n$.
 %   Since $S(0, a) = 0 = T(0, a)$,
 %   the basis of our induction holds true.
 %   It remains to check the inductive step.
 %   Straightforward computations show that the bivariate polynomial function $T$ satisfy the identity
 %   $$
 %   %y T(x, y) -  T(x + 1, y) = \frac{x(x - 1)}{2}  {(y - 1)}^2 - 1 
 %   T(n + 1, a) = \left( a T(n, a) + 1 \right) - \frac{n(n - 1)}{2}  {(a - 1)}^2 \,, 
 %   $$
 %   whence 
 %   $$
 %   %D(n + 1) =  a D(n) + \frac{n(n - 1)}{2} {(a - 1)}^2 \, .
 %   S(n + 1, a) - T(n + 1, a) =  a \left( S(n, a) - T(n, a) \right) + \frac{n(n - 1)}{2} {(a - 1)}^2 \, .
 %   $$

 %   $$
 %   S(n, a) - T(n, a) = {(a - 1)}^2 \sum_{k = 0}^{n - 1}  \frac{k(k - 1)}{2} a^{n - k}  
 %   $$
 %   %Therefore, if ${(a - 1)}^2$ divides $D(n)$ then ${(a - 1)}^2$ divides $D(n + 1)$.
 %   Therefore, if ${(a - 1)}^2$ divides $S(n, a) - T(n, a)$ then ${(a - 1)}^2$ divides $S(n + 1, a) - T(n + 1, a)$.
 % \end{proof}
$$
10^9 < 9^{10}
$$

$$
\frac{10^9}{9^{10}}
= \frac{1}{9} \left( \frac{10}{9} \right)^9
= \frac{1}{9} \left( 1 + \frac{1}{9} \right)^9
< \frac{1}{9} e < 1
$$

$$
100^{99} < 99^{100}
$$

$$
\frac{100^{99}}{99^{100}}
= \frac{1}{99} \left( \frac{100}{99} \right)^{99}
= \frac{1}{99} \left( 1 + \frac{1}{99} \right)^{99}
< \frac{1}{99} e < 1
$$


  Theorem~\ref{thm:2-adic-S} trivially holds true in the case $a = 1$ because $S(n, 1) = n$ for every $n \in \bN$.
  Hence, we may assume $a \ne 1$.
  For each $n \in \bN$, put $\mu(n) = \nu_2(S(n, a))$ and
  $t(n)$
  Let $n \in \bN$.
  The crux is to prove
  \begin{equation} \label{eq:nu-S2n-2Sn}
    \nu_2(S(2n, a)) = \nu_2(S(n, a)) + 1 
  \end{equation}
  and that $S(2n + 1, a)$ is odd.
  % \begin{equation}  \label{eq:nu-S2n+1-0}
  %   \nu_2(S(2n + 1, a)) = 0 \, .
  % \end{equation}
  Put 
  $$
  t = 1 + \frac{a - 1}{2} S(n, a) \, . 
  $$
  Since Equation~\eqref{eq:ak-1-a-1-Ska} holds true for $k \in \{ n, 2n \}$, straightforward computations yield
  \begin{align*}
    (a - 1) \left( S(2 n, a) - 2 S(n, a) \right) & = (a^{2n} - 1) - 2(a^n - 1) \\
                                                 & = {(a^n - 1)}^2 \\
                                                 & = {(a - 1)}^2 \left( S(n, a) \right)^2\,.
  \end{align*}
  It follows 
  $$
  S(2 n, a) - 2 S(n, a) = (a - 1) \left(  S(n, a) \right)^2 \,, 
  $$
  or equivalently,
  $$
  S(2 n, a) = 2 t S(n, a) \, .
  $$
  % S(2 n, a) = 2 \left(1 + \frac{a - 1}{2}  S(n, a)  \right) S(n, a) \, .
  Besides,  $t$ is an odd integer because $(a - 1) \mathbin{/} 2$ is an even integer.
  Therefore, Equation~\eqref{eq:nu-S2n-2Sn} holds true.
%  Let us now prove Equation~\eqref{eq:nu-S2n+1-0}. 
  In particular, Equation~\eqref{eq:nu-S2n-2Sn} ensures that $S(2n, a)$ is even, whence
  $$
  S(2n + 1, a) = 1 + S(2n, a) a
  $$
  is odd. %, and thus Equation~\eqref{eq:nu-S2n+1-0} holds true.
  Let us now conclude the proof of Theorem~\ref{thm:2-adic-S}.
  Our task is to prove $\nu_2(S(n, a)) = \nu_2(n)$.
  Since $S(0, a) = 0$, we may assume $n \ne 0$.
  
  


% the $p$-adic valuation of $p^k q$ is equal to $k$ if, and only if, $p$ does not divide~$q$.
$$
\nu_p(nn') = \nu_p(n) + \nu_p(n') 
$$

$$
\nu_p(n + n') \ge \inf \left\{ \nu_p(n), \nu_p(n') \right\} 
$$

$$
\nu_p(n + n') \ge 1 + \inf \left\{ \nu_p(n), \nu_p(n') \right\} 
$$
implies $\nu_p(n) = \nu_p(n')$

$$
\nu_p(p n ) = \nu_p(n) + 1
$$
 \begin{proof}
      % Put
   % $$
   % g(\ttx) = 1 - r  \ttx - \frac{r(r - 1)}{2} \ttx^2
   % $$
   % and
   % $$
   % f(\ttx) =  {(1 + \ttx)}^r - g(\ttx)1 - r  \ttx - \frac{r(r - 1)}{2} \ttx^2
   % $$
   % Since $S(r, 1 + 2r) = r$ for each $r \in \{ 0, 1 \}$, we may assume $r \ge 2$.
   Put
   $$
   f(\ttx) = \sum_{k = 3}^r \binom{r}{k} \ttx^{k - 3}
   $$
   and $q =  r - 1  +  4 f(2r)$.
   By construction, $f(\ttx)$ lies in $\bZ[\ttx]$, $q$ is an integer, and the binomial theorem ensures 
   $$
   {(1 + \ttx)}^r = 1 + r \ttx + \frac{r(r - 1)}{2} \ttx^2 + \ttx^3 f(\ttx)  \, .
   $$
   It follows
   $$
   S(r, 1 + 2 \ttx) = \frac{{(1 + 2 \ttx)}^r - 1}{2 \ttx} = r + r(r - 1) \ttx + 4 \ttx^2 f(2 \ttx) \,,
   $$
   and consequently, $S(r, 1 + 2r) = r + q r^2$.
 \end{proof}

\begin{proof}
  Since $0$ divides $0 = S(0, a)$,  we may assume $m \ge 1$ and proceed by induction on $\Omega(m)$.
  Since  $\Omega(m) = 0$ is equivalent to $m = 1$ and since $1$ divides every integer,
  the basis of the induction holds.
  Let us now assume  $\Omega(m) \ge 1$.
  Then, there exist $m'$, $p \in \bN$  such that  $p$ is prime and $m = m' p$.
  Since every prime factor of $m'$ is a prime factor of $m$ and since every prime factor of , the induction hypothesis ensures that $n$ divides $S(n, a)$.
  Moreover, if $p$ is odd then 
  $$
  S(n n', a) = S(n, a) S(n', a^n)
  $$
\end{proof}


$p$ divides $f(0)$

$$
\frac{f(p) - f(0)}{p}
$$


 \begin{exercise}  \label{exo:n-odd-carre}
   Let $a \in \bZ$ and let $p \in \bN$.
   Assume that $p$ is odd and that $p$ divides $a  - 1$.
   Prove $S(p, a) \equiv p \pmod {p^2}$.
 \end{exercise}
 
 \begin{proof}
   Let us first consider the case where $a - 1$ is even.
   Then, $2p$ divides $a - 1$, and thus Theorem~\ref{thm:n-even-ncarre} yields the desired congruence.
   Let us now consider the case where $a - 1$ is odd.
   Then, $a + p^2 - 1$ is even, and thus the previous discussion ensures $S(n, a + p^2) \equiv p \pmod {p^2}$.
   Besides, $S(p, a + \ttx)$ is a polynomial with integer coefficients,
   whence $S(p, a) \equiv S(p, a + p^2) \pmod{p^2}$.
   Therefore, the desired congruence  holds true.
 \end{proof}
 

 \begin{theorem} \label{thm:2p-p-adic}
   Let $a \in \bZ$ and let $p$ be a prime such that $2p$ divides $a - 1$.
   For each $n \in \bNast$, the $p$-adic order of $S(n, a)$ is equal to that of~$n$.
 \end{theorem}

 \begin{proof}
   Let $\nu$ denote the $p$-adic order of~$n$.
   We proceed by induction on $\nu(n)$.
   Theorem~\ref{th:div-a-div-S-div-n} ensures that $\nu(n) = 0$ is equivalent to  $\nu(S(n, a)) = 0$,
   so the basis of our induction holds.
   Let us now assume $\nu(n) \ge 1$.
   Put $m =  p^{-1} n$.
   The induction hypothesis yields
   $$
   \nu(S(m, a)) = \nu(n) - 1 \, .
   $$
%   because $\nu(n p^{-1})$ is one less than $\nu(n)$ and every factor of $a - 1$ is a factor of $a^p - 1$.
   Moreover, we have 
   $$
   S(n, a) = S(p, a^m) S(m, a) 
   $$
   and Theorem~\ref{thm:n-odd-ncarre} yields 
   $$\nu(S(p, a^m)) = 1 \, .$$
   It follows
   $$
   \nu(S(n, a))
   = \nu(S(p, a^m)) + \nu(S(m, a))
   = 1 + (\nu(n) - 1)
   = \nu(n) \,, 
   $$
   as desired.  
 \end{proof}

 \begin{exercise} \label{exo:p-p-adic}
   Let $a \in \bZ$ and let $p$ be a odd prime factor of $a - 1$.
   Prove that for each $n \in \bN$, the $p$-adic order of $S(n, a)$ is equal to that of~$n$.
 \end{exercise}

 \begin{proof}
   Let us first consider the case were $a - 1$ is even.
   Then, $2p$ divides $a - 1$,
   and thus Theorem~\ref{thm:2p-p-adic} ensures that the desired property holds true.
   Let us now consider the case where $a - 1$ is odd.
   Let $\nu$ denote the $p$-adic order of $n$ and let $a' = a + p^{\nu + 1}$.
   By construction, $a' - 1$ is even,
   so the previous discussion ensures that the $p$-adic order of $S(n, a')$ equals~$\nu$.
   Besides, $S(n, a + \ttx)$ is a polynomial with integer coefficients,
   whence $S(n, a') \equiv S(n, a) \pmod{p^{\nu + 1}}$.
   Therefore, the $p$-adic order of $S(n, a)$ equals~$\nu$.   
 \end{proof}
 
 

   %  \item Let $\nu$ denote the $p$-adic order of $n$.
   %   Prove $p^{-\nu} S(n, a) \equiv S(p^{- \nu} n, a) \pmod{p}$. 
   % \end{enumerate} 


$$
S(n, a) - S(n', a) = {(a + 1)}^n S(n - n', a)
$$

$$
d = \gcd(m, n)
$$

$$
S(n, a) = S(d, a^{n/d}) S(n/d, a)
$$

 
$$
\frac{S(n, a)}{d}  = \frac{1}{p} \frac{S(n, a)}{d} d
% \frac{S(d, a^{n\mathbin{/}d})}{d}
% \frac{S(n\mathbin{/}d, a)}{m \mathbin{/} d} 
$$

% Assume that $a - 1$ is odd.
% Then, $m$ is odd and $a + m - 1$ is even.
% Therefore, for each prime factor $p$ of $m$, $2p$ divides $a + m - 1$.

% Assume that $a - 1$ is even and that $4$ does not divide $a - 1$.
% The, for each prime factor $p$ of $m$, $2p$ divides $a + m - 1$.

   
