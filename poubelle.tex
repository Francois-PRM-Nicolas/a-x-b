
%%% Local Variables:
%%% mode: latex
%%% TeX-master: "preambule"
%%% End:



 \begin{proof}
      % Put
   % $$
   % g(\ttx) = 1 - r  \ttx - \frac{r(r - 1)}{2} \ttx^2
   % $$
   % and
   % $$
   % f(\ttx) =  {(1 + \ttx)}^r - g(\ttx)1 - r  \ttx - \frac{r(r - 1)}{2} \ttx^2
   % $$
   % Since $S(r, 1 + 2r) = r$ for each $r \in \{ 0, 1 \}$, we may assume $r \ge 2$.
   Put
   $$
   f(\ttx) = \sum_{k = 3}^r \binom{r}{k} \ttx^{k - 3}
   $$
   and $q =  r - 1  +  4 f(2r)$.
   By construction, $f(\ttx)$ lies in $\bZ[\ttx]$, $q$ is an integer, and the binomial theorem ensures 
   $$
   {(1 + \ttx)}^r = 1 + r \ttx + \frac{r(r - 1)}{2} \ttx^2 + \ttx^3 f(\ttx)  \, .
   $$
   It follows
   $$
   S(r, 1 + 2 \ttx) = \frac{{(1 + 2 \ttx)}^r - 1}{2 \ttx} = r + r(r - 1) \ttx + 4 \ttx^2 f(2 \ttx) \,,
   $$
   and consequently, $S(r, 1 + 2r) = r + q r^2$.
 \end{proof}

\begin{proof}
  Since $0$ divides $0 = S(0, a)$,  we may assume $m \ge 1$ and proceed by induction on $\Omega(m)$.
  Since  $\Omega(m) = 0$ is equivalent to $m = 1$ and since $1$ divides every integer,
  the basis of the induction holds.
  Let us now assume  $\Omega(m) \ge 1$.
  Then, there exist $m'$, $p \in \bN$  such that  $p$ is prime and $m = m' p$.
  Since every prime factor of $m'$ is a prime factor of $m$ and since every prime factor of , the induction hypothesis ensures that $n$ divides $S(n, a)$.
  Moreover, if $p$ is odd then 
  $$
  S(n n', a) = S(n, a) S(n', a^n)
  $$
\end{proof}


$p$ divides $f(0)$

$$
\frac{f(p) - f(0)}{p}
$$


 \begin{exercise}  \label{exo:n-odd-carre}
   Let $a \in \bZ$ and let $p \in \bN$.
   Assume that $p$ is odd and that $p$ divides $a  - 1$.
   Prove $S(p, a) \equiv p \pmod {p^2}$.
 \end{exercise}
 
 \begin{proof}
   Let us first consider the case where $a - 1$ is even.
   Then, $2p$ divides $a - 1$, and thus Theorem~\ref{thm:n-even-ncarre} yields the desired congruence.
   Let us now consider the case where $a - 1$ is odd.
   Then, $a + p^2 - 1$ is even, and thus the previous discussion ensures $S(n, a + p^2) \equiv p \pmod {p^2}$.
   Besides, $S(p, a + \ttx)$ is a polynomial with integer coefficients,
   whence $S(p, a) \equiv S(p, a + p^2) \pmod{p^2}$.
   Therefore, the desired congruence  holds true.
 \end{proof}
 

 \begin{theorem} \label{thm:2p-p-adic}
   Let $a \in \bZ$ and let $p$ be a prime such that $2p$ divides $a - 1$.
   For each $n \in \bNast$, the $p$-adic order of $S(n, a)$ is equal to that of~$n$.
 \end{theorem}

 \begin{proof}
   Let $\nu$ denote the $p$-adic order of~$n$.
   We proceed by induction on $\nu(n)$.
   Theorem~\ref{th:div-a-div-S-div-n} ensures that $\nu(n) = 0$ is equivalent to  $\nu(S(n, a)) = 0$,
   so the basis of our induction holds.
   Let us now assume $\nu(n) \ge 1$.
   Put $m =  p^{-1} n$.
   The induction hypothesis yields
   $$
   \nu(S(m, a)) = \nu(n) - 1 \, .
   $$
%   because $\nu(n p^{-1})$ is one less than $\nu(n)$ and every factor of $a - 1$ is a factor of $a^p - 1$.
   Moreover, we have 
   $$
   S(n, a) = S(p, a^m) S(m, a) 
   $$
   and Theorem~\ref{thm:n-odd-ncarre} yields 
   $$\nu(S(p, a^m)) = 1 \, .$$
   It follows
   $$
   \nu(S(n, a))
   = \nu(S(p, a^m)) + \nu(S(m, a))
   = 1 + (\nu(n) - 1)
   = \nu(n) \,, 
   $$
   as desired.  
 \end{proof}

 \begin{exercise} \label{exo:p-p-adic}
   Let $a \in \bZ$ and let $p$ be a odd prime factor of $a - 1$.
   Prove that for each $n \in \bN$, the $p$-adic order of $S(n, a)$ is equal to that of~$n$.
 \end{exercise}

 \begin{proof}
   Let us first consider the case were $a - 1$ is even.
   Then, $2p$ divides $a - 1$,
   and thus Theorem~\ref{thm:2p-p-adic} ensures that the desired property holds true.
   Let us now consider the case where $a - 1$ is odd.
   Let $\nu$ denote the $p$-adic order of $n$ and let $a' = a + p^{\nu + 1}$.
   By construction, $a' - 1$ is even,
   so the previous discussion ensures that the $p$-adic order of $S(n, a')$ equals~$\nu$.
   Besides, $S(n, a + \ttx)$ is a polynomial with integer coefficients,
   whence $S(n, a') \equiv S(n, a) \pmod{p^{\nu + 1}}$.
   Therefore, the $p$-adic order of $S(n, a)$ equals~$\nu$.   
 \end{proof}
 
 

   %  \item Let $\nu$ denote the $p$-adic order of $n$.
   %   Prove $p^{-\nu} S(n, a) \equiv S(p^{- \nu} n, a) \pmod{p}$. 
   % \end{enumerate} 


$$
S(n, a) - S(n', a) = {(a + 1)}^n S(n - n', a)
$$

$$
d = \gcd(m, n)
$$

$$
S(n, a) = S(d, a^{n/d}) S(n/d, a)
$$

 
$$
\frac{S(n, a)}{d}  = \frac{1}{p} \frac{S(n, a)}{d} d
% \frac{S(d, a^{n\mathbin{/}d})}{d}
% \frac{S(n\mathbin{/}d, a)}{m \mathbin{/} d} 
$$

% Assume that $a - 1$ is odd.
% Then, $m$ is odd and $a + m - 1$ is even.
% Therefore, for each prime factor $p$ of $m$, $2p$ divides $a + m - 1$.

% Assume that $a - 1$ is even and that $4$ does not divide $a - 1$.
% The, for each prime factor $p$ of $m$, $2p$ divides $a + m - 1$.

   
