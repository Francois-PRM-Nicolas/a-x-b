% -*- mode: latex; eval: (flyspell-mode 1); ispell-local-dictionary: "american"; TeX-master: "preambule.tex"; -*-

\begin{theorem} \label{thm:cyclic-iteration}
  Let $X$ be a finite set,
  let $E\colon \bN \to X$, and
  let $\sigma\colon X \to X$ be such that
  $E(n + 1) = \sigma(E(n))$ for every $n \in \bN$.
  Let $m$ denote the cardinality of~$X$.
  The following four assertions are equivalent:
  \begin{enumerate}
\item \label{ass:cyclic-iteration:0-m}
  The restriction of $E$ to $\seg{0}{m - 1}$ is bijective and equality $E(m) = E(0)$ holds true.
%\item \label{ass:cyclic-iteration:distinct}
%  For each $n \in \seg{1}{m - 1}$, $E(n)$ is distinct from $E(m)$.
\item \label{ass:cyclic-iteration:1-m}
  The restriction of $E$ to $\seg{1}{m}$ is bijective.
\item \label{ass:cyclic-iteration:surj-bij}
  $E$ is surjective and $\sigma$ is bijective.
\item \label{ass:cyclic-iteration:cycl}
  For every $Y \subseteq X$,
  $\sigma(Y) = Y$ implies $Y \in \{ \emptyset, X \}$. 
  \end{enumerate}
\end{theorem}

\begin{proof}
  Clearly,
  assertion~\ref{ass:cyclic-iteration:0-m} implies assertion~\ref{ass:cyclic-iteration:distinct}
  and 
  assertion~\ref{ass:cyclic-iteration:1-m} implies assertion~\ref{ass:cyclic-iteration:surj-bij}.
  %assertion~\ref{ass:cyclic-iteration:0-m} implies assertion~\ref{ass:cyclic-iteration:distinct} and 
  %assertion~\ref{ass:cyclic-iteration:1-m} implies assertion~\ref{ass:cyclic-iteration:surj-bij}.
  
  $\ref{ass:cyclic-iteration:distinct} \implies \ref{ass:cyclic-iteration:1-m}$.
  Assume that  assertion~\ref{ass:cyclic-iteration:1-m} holds false.
  Then, the restriction of $E$ to $\seg{1}{m}$ is not injective,
  and thus there exist $p$, $q \in \seg{1}{m}$ such that $p < q$ and $E(p) = E(q)$.
  A straightforward induction on $n$ shows $E(p + n) = E(q + n)$ for every $n \in \bN$.
  In the particular case where $n = m - q$,
  we obtain $E(m + p - q) = E(m)$.
  Therefore, assertion~\ref{ass:cyclic-iteration:1-m} holds false.

  $\ref{ass:cyclic-iteration:surj-bij} \implies \ref{ass:cyclic-iteration:cycl}$.
  Assume that assertion~\ref{ass:cyclic-iteration:surj-bij} holds true.
  Let $Y \subseteq X$ be such that $\sigma(Y) = Y$ and $Y \ne \emptyset$.
  Our task is to prove $Y = X$. 
  Since $E$ is surjective, there exists $k \in \bN$ such that $E(k) \in Y$.
  Since $Y$ is closed under $\sigma$,
  a straightforward induction on $n$ shows $E{(k + n)} \in Y$ for each $n \in \bN$,
  whence
  $$
  E(\bZ_{\ge k}) \subseteq Y \, .
  $$
  Since $Y$ is closed under $\sigma^{-1}$, 
  a straightforward induction on $n$ shows $E{(k - n)} \in Y$ for each $n \in \seg{0}{k}$,
  whence
  $$
  E(\seg{0}{k}) \subseteq Y \, .
  $$
  % Besides, we have 
  % $$
  % \bZ_{\ge 0} =  \seg{0}{k} \cup \bZ_{\ge k} \, . 
  % $$
  It follows
  $$
  X = E(\bN) = E(\seg{0}{k}) \cup E(\bZ_{\ge k}) \subseteq Y \,, 
  $$
  and thus assertion~\ref{ass:cyclic-iteration:cycl} holds true.

  $\ref{ass:cyclic-iteration:cycl} \implies \ref{ass:cyclic-iteration:0-m}$.
  Since the restriction of $E$ to $\seg{0}{m}$ is not injective,
  there exist $p$, $q \in \seg{0}{m}$ such that $p < q$ and $E(p) = E(q)$.
  By construction, $\sigma$ maps $E(\seg{p}{q - 1})$ onto itself.
  Assume that assertion~\ref{ass:cyclic-iteration:cycl} holds true.
  Then, we have $E(\seg{p}{q - 1}) = X$.
  In particular, the cardinality of $\seg{p}{q - 1}$ is not less than~$m$.
  Besides, $\seg{p}{q - 1}$ is a subset of $\seg{0}{m - 1}$.
  It follows $\seg{p}{q - 1} = \seg{0}{m - 1}$, or equivalently, $(p, q) = (0, m)$.
  Therefore, assertion~\ref{ass:cyclic-iteration:0-m} holds true. 
\end{proof}

Let $X$ be a set and let $g\colon\bN \to X$.
We say that $g$ is an \emph{orbit} if the following two equivalent conditions are met:
%The following three assertions are equivalent:
\begin{enumerate}
\item There exists $f \colon X \to X$  such that $g(n + 1) = f(g(n))$ for every $n \in \bN$.
%\item For every $p$, $q \in \bN$, $g(p) = g(q)$ implies $g(p + 1) = g(q + 1)$.
\item For every $n$, $p$, $q \in \bN$, $g(p) = g(q)$ implies $g(n + p) = g(n + q)$.
\end{enumerate}
The following theorem summarizes what may happen


q + k p + n  = 2 (p + q)

q = 
$n + jp + q = 2(n + k p + q)$

0 = (2k - j) p + n + q
k = 0
j = q
0 = - q p + n + q
n = q (p - 1)

q (p - 1) + q p + q = 
\begin{theorem} \label{thm:eventually-periodic}
  Let $X$ be a set and let $g\colon \bN \to X$ be non-injective.
  Let $P$ denote the set of those $p \in \bNast$ for which there exists $n \in \bN$ such that $g(n + p) = g(n)$.
  Let $Q$ denote the set of those $q \in \bNast$ such that $g(q + p) = g(q)$.
  There exists $p \in \bNast$ and $q \in \bN$ such that $g$ is injective on $\seg{0}{p + q}$ and $g(p + q) = g(q)$.
    There exists $f\colon X \to X$ such that $g(n + 1) = f(g(n))$ for every $n \in \bN$ if, and only if,
   %The function $g$ is  a non-injective orbit if, and only if,
  there exist $p \in \bNast$ and $q \in \bN$ such that
  \begin{itemize}
    \item 
      $g$ is injective on $\sego{0}{p + q}$ and
     \item 
       equality $g(n + p + q) = g(n + q)$ holds true for every $n \in \bN$.
   \end{itemize}
     % $g$ is injective on $\sego{0}{p + q}$ and
  % equality  $g(n + p + q) = g(n + q)$ holds true for every $n \in \bN$.
%   The following two assertions are equivalent:
%   \begin{enumerate}
%   \item \label{ass:eventually-periodic:f}
%     $g$ is a non-injective orbit.
% %    there exists $f \colon X \to X$ such that
%     % \begin{equation} \label{eq:eventually-periodic:fg}
%     % g(n + 1) = f(g(n))
%     % \end{equation}  for every $n \in \bN$.
%     %$g(n + 1) = f(g(n))$ for every $n \in \bN$.
  % \item \label{ass:eventually-periodic:q}
  %   There exist $p \in \bNast$ and $q \in \bN$ such that
  %   $g$ is injective on $\sego{0}{p + q}$ and
  %   $g(n + p + q) = g(n + q)$ for every $n \in \bN$.
  % \end{enumerate}
\end{theorem}

\begin{proof}
  $\ref{ass:eventually-periodic:f} \implies \ref{ass:eventually-periodic:q}$.
  Put
  $$R = \left\{ n \in \bN : g(n) \in g(\sego{0}{n}) \right\} \, .$$
  Assume that assertion~\ref{ass:eventually-periodic:f} holds true.
  Since $g$ is not injective, $R$ is non-empty;
  let $r$ denote the least element of~$R$.
  On the one hand, $r$ is an element of $R$,
  whence there exists $q \in \sego{0}{r}$ such that $g(r) = g(q)$.
  Since Equation~\eqref{eq:eventually-periodic:fg} holds true for every $n \in \bN$,
  a straightforward induction on $n$ shows that equality $g(n + r) = g(n + q)$ holds true for every $n \in \bN$.
  On the other hand, $\sego{0}{r}$ does not intersect $R$,
  so $g$ is injective on $\sego{0}{r}$.
  Therefore, assertion~\ref{ass:eventually-periodic:f} holds true with $p = r - q$.
% V  
%   Since $g$ is not injective on $\seg{0}{m}$, there exist $r$, $s \in \bZ$ such that
%   $0 \le r < s \le m$ and
%   $g(r) \ne g(s)$.
%   Then, a straightforward induction on $n$ shows $g(n + r) = g(n + s)$ for every $n \in \bN$,
%   and thus assertion~\ref{thm:eventually-periodic:f} holds true with $q = m - s + r$.
   
  $\ref{ass:eventually-periodic:q} \implies \ref{ass:eventually-periodic:f}$.
  Assume that assertion~\ref{ass:eventually-periodic:q} holds true.
  Since $g(p + q) = g(q)$, $g$ is not injective.
  Since $g$ is injective on $\sego{0}{p + q}$,
  there exists $f\colon X \to X$ such that $f(g(n)) = g(n + 1)$ for each $n \in \sego{0}{p + q}$
  (note that $X \ne g(\sego{0}{p + q})$ may happen, so $f$ is not uniquely defined).
  Since  $g(n) = g(n - p) $ and $g(n + 1) = g(n - p + 1)$ for each $n \in \bZ_{\ge p + q}$,
  a straightforward induction on $n$ shows that
  Equation~\eqref{eq:eventually-periodic:fg} holds true for every $n \in \bZ_{\ge p + q}$.
  Therefore, assertion~\ref{ass:eventually-periodic:f} holds true.
 \end{proof} 

Let $X$ be a set, let $p \in \bZ$, and let $F \colon \bZ \to X$.
We say that $p$ is a \emph{period} of $F$ if equality $F(n + p)  = F(n)$ holds true for every $n \in \bZ$.
%The set of all periods of $F$ is an additive subgroup of~$\bZ$.
%If $p$ is period of $F$ 
In the case where $p$ is non-zero, 
$p$ is a period of $F$ if, and only if, equality $F(n) = F(n \bmod p)$ holds true for every $n \in \bZ$.

\begin{theorem} \label{thm:def-cyclic-enumeration} 
  Let $X$ be a finite set and let $F \colon \bZ \to X$.
  Let $m$ denote the cardinality of~$X$.
  The following three assertions are equivalent:
  \begin{enumerate}
  \item \label{ass:def-cyclic-enumeration:surj-periodic}
    $F$ is surjective and $m$ is a period of~$F$.
  \item \label{ass:def-cyclic-enumeration:bij-per}
    For each $n \in \bZ$, $F$ induces a bijection from $\sego{n}{n + m}$ onto~$X$.
  \item \label{ass:def-cyclic-enumeration:cong}
    For every $n$, $n' \in \bZ$ such that $F(n) = F(n')$, $m$ divides $n - n'$.
  \end{enumerate} 
\end{theorem}

   %There exists a function $\sigma\colon X \to X$ such that $F(n + 1) = \sigma(F(n))$  for every $n \in \bZ$.
   %For every $Y \subseteq X$, $\sigma(Y) = Y$ implies $Y \in \{ \emptyset, X \}$.

\begin{proof}
  $\ref{ass:def-cyclic-enumeration:bij-per} \implies \ref{ass:def-cyclic-enumeration:surj-periodic}$.
  Assume that assertion~\ref{ass:def-cyclic-enumeration:bij-per} holds true.
  Let $n \in \bN$.
  Since the cardinality of $\seg{n}{n + m}$ is equal to $m + 1$,
  $F$ is not injective on that set,
  and thus there exist $p$, $q \in \seg{n}{n + m}$ such that $p < q$ and $F(p) = F(q)$.
  The injectivity of $F$ on $\sego{n}{n + m}$ precludes $\{ p, q \} \subseteq \sego{n}{n + m}$,
  whence $q = m + n$;
  the injectivity of $F$ on $\seg{n + 1}{n + m}$ precludes $\{ p, q \} \subseteq \seg{n + 1}{n + m}$,
  whence $p = n$.
  It follows $F(n + m) = F(n)$, and thus $m$ is a period of~$F$.
  Therefore, assertion~\ref{ass:def-cyclic-enumeration:surj-periodic} holds true.
  
  
  $\ref{ass:def-cyclic-enumeration:cong} \implies \ref{ass:def-cyclic-enumeration:bij-per}$.
  For every $n$, $n' \in \bZ$ such that $m$ divides $n - n'$, $n \ne n'$ is equivalent to $\abs{n - n'} \ge m$.
  Therefore, assertion~\ref{ass:def-cyclic-enumeration:cong} implies assertion~\ref{ass:def-cyclic-enumeration:bij-per}.

  $\ref{ass:def-cyclic-enumeration:surj-periodic} \implies \ref{ass:def-cyclic-enumeration:cong}$.
  Assume that assertion~\ref{ass:def-cyclic-enumeration:surj-periodic} holds true.
  Since $F(n) = F(n \bmod m)$ for every $n \in \bZ$ and since $F$ is surjective,
  $F$ is induces a bijection from $\sego{0}{m}$ onto~$X$.
  Hence, for every $n$, $n' \in \bZ$, the following four conditions are equivalent:
  $F(n) = F(n')$,
  $F(n \bmod m) = F(n' \bmod m)$,
  $n \bmod m = n' \bmod m$, and
  $m$ divides $n - n'$.
  Therefore, assertion~\ref{ass:def-cyclic-enumeration:cong} holds true.
\end{proof}

Let the notation be as in Theorem~\ref{thm:def-cyclic-enumeration}.
We say that $F$ is a \emph{cyclic enumeration} (of $X$)
if the three equivalent assertions of Theorem~\ref{thm:def-cyclic-enumeration} hold true.

%The restriction to $\bN$ of any cyclic enumeration is also called a cyclic enumeration. 


\begin{theorem}
  Let $X$ be a finite set and let $F\colon \bN \to X$.
  There exists a cyclic enumeration $F' \colon \bZ \to F(X)$ such that $F$ and $F'$ coincide on $\bN$
  if, and only if, the following two conditions are met:
  \begin{enumerate}
  \item There exists $\sigma \colon X \to X$ such that $F(n + 1) = \sigma(F(n))$ for every $n \in \bN$.
  \item There exists $n_0 \in \bNast$ such that $F(n_0) = F(0)$.
  \end{enumerate} 
\end{theorem} 






 \begin{theorem}
   Let the notation be as in Theorem~\ref{thm:def-cyclic-enumeration}.
   Assume that 
   there exists $n_0 \in \bNast$ such that $g(n_0) = g(0)$ and 
   that there exists $f\colon X \to X$ such that $g(n + 1) = f(g(n))$ for every $n \in \bN$.
   Then, $g$ is a cyclic enumeration. 
\end{theorem}


% \begin{theorem}
%   In the notation of Theorem~\ref{thm:def-cyclic-enumeration}, 
    The restriction of $g$ to $\bNast$ is surjective and  
    there exists $f\colon X \to X$ such that $g(n + 1) = f(g(n))$ for every $n \in \bN$.
  % \item \label{ass:def-cyclic-enumeration:cong}
  %   For every $n$, $n' \in \bN$ such that $g(n) = g(n')$, $m$ divides $n - n'$.
%   $g$ is a cyclic enumeration if and only if, the following two conditions are met:
%   \begin{enumerate}
%   \item there exists $n_0 \in \bNast$ such that $g(n_0) = g(0)$ and 
%   \item there exists $f\colon X \to X$ such that $g(n + 1) = f(g(n))$ for every $n \in \bN$.
%   \end{enumerate}
% \end{theorem}

\begin{theorem}
  Let the notation be as in Theorem~\ref{thm:def-cyclic-enumeration}.
  Assume that for every $n$, $n' \in \bN$ such that $g(n + n') = g(n')$, $m$ divides $n$.
  Then, $g$ is a cyclic enumeration.
\end{theorem}
% A \emph{cyclic derangement} of $X$ is a function $g \colon X \to X$ such that
% for every $Y \subseteq X$, $g(Y) = Y$ implies $Y \in \{ \emptyset, X \}$.

\begin{theorem}
  Let $X$ be a finite set,
  let $f\colon \bN \to X$, and
  let $g\colon X \to X$ be such that
  $f(n + 1) = g(f(n))$ for every $n \in \bN$.
  The following three assertions are equivalent:
  \begin{enumerate}
  \item $f$ is a cyclic enumeration of~$X$.
  \item $f$ is surjective and $\sigma$ is bijective.
  \item $g$ is a cyclic derangement of~$X$.
  \end{enumerate} 
\end{theorem}

  $\ref{ass:def-cyclic-enumeration:surj-f} \implies  \ref{ass:def-cyclic-enumeration:bij-per}$.
  Assume that assertion~\ref{ass:def-cyclic-enumeration:surj-f} holds true.
  Since $\bN + 1 = \bNast$,
  $f$ maps $g(\bN)$ onto $g(\bN + 1) = g(\bNast) = X$,
  whence $f$ is bijective.
  Let $p$, $q \in \bN$ be such that $p < q$ and $g(p) = g(q)$.
  Put $Y = g(\sego{p}{q})$.
  By construction, both $f$ and $f^{-1}$ map $Y$ onto itself.
  Since $g(p) \in Y$ and since
  $$g(n - 1) = f^{-1}(g(n))$$
  for every $n \in \bNast$,
  a straightfoward induction on $n$ shows that 
  $g(p - n)$ belongs to $Y$ for every $n \in \sego{0}{p}$.
  In particular, $g(1)$ belongs to $Y$, and thus 
  a straightfoward induction on $n$ shows that
  $g(n)$ belongs to $Y$ for every $n \in \bNast$.
  It follows $Y = X$, and consequently, $q - p \ge m$.
  Hence, assertion~\ref{ass:def-cyclic-enumeration:bij-per} holds true.