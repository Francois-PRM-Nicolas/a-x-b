% -*- mode: latex; eval: (flyspell-mode 1); ispell-local-dictionary: "american"; TeX-master: "preambule"; -*-

\section{Additional material}

% \begin{theorem} \label{thm:Frobenius}
%   Let $k \in \bZ$ and let $p \in \bN$.
%   If $1 \le k < p$ and if $p$ is prime then $p$ divides the binomial cofficient coefficient of $\ttx^k$ in ${(1 + \ttx)}^p $ is a multiple of~$p$.
% \end{theorem}

% \begin{proof}
%   For each $k \in \bN$, let $a_k$ denote the coefficient of $\ttx^k$ in  ${(1 + \ttx)}^p - 1 - \ttx^p$:
%   $$
%   a_k =  \frac{p!}{k!{(p - k)}!}
%   $$
  
%   The binomial theorem ensures
%   $$
%   a_k = 
%   $$
%   If $k = 0$ or if $k \ge p$ then $a_k = 0$.
%   Let $k$ be an integer such that $1 \le k < p$.
%   For each $n \in \bN$,
%   the set of all prime factors of $n!$ is equal to the set of those primes that are not greater than $n$.
%   Therefore, every prime factor of $k!{(p - k)}!$ is less than~$p$.
%   Besides, $p$ divides $p (p - 1)! = p!$.
%   Therefore, $p$ divides the binomial coefficient 

% \end{proof}

% The most famous consequence of Theorem~\ref{thm:Frobenius} is:

% \begin{exercise}
%   Let $p$ be a prime and let $R$ be a commutative ring with characteristic~$p$: $p R = \{ 0 \}$.
%   Prove that ${(x+ y)}^p =  x^p + y^p $ for every $x$, $y \in R$.
% \end{exercise}

\subsection{Powers of $2$}

For each $n \in \bZ$, let $\nu(n)$ denote the $2$-adic valuation of $n$:
$\nu(n) \in \bN \cup \{ \infty \}$,
$\nu(n) = \infty$ is equivalent to $n = 0$, and
if $n \ne 0$ then  $2^{- \nu(n)} n$ is an odd integer.
%Note that $\gcd(m, 2^n)  = 2^{\min \{ \nu(m), n \}}$ for every $m \in \bZ$ and every $n \in \bN$.
The first part of Theorem~\ref{thm:val-adic} can be restated as follows:

\begin{theorem} \label{thm:val-adic-2-alt} 
  Let $a \in \bZ$ and let $n \in \bN$.
  If $4$ divides $a - 1$ then $\nu(S(n, a)) = \nu(n)$.
\end{theorem}



\begin{theorem} \label{thm:order-un-mod-4} 
  Let $a \in \bZ$ be such that $4$ divides $a - 1$ and let $m \in \bN$.
  The multiplicative order of $a$ modulo $2^m$ is equal to
  $$
  \frac{2^m}{\gcd(a - 1, 2^m)} \,.
  $$
\end{theorem}

\begin{proof}
  Put $d = \gcd(a - 1, 2^m)$ and $k = 2^m d^{-1}$.
  Let $n \in \bN$.
  Our task is to prove that the following four assertions are equivalent:
  \begin{enumerate}
    \item $2^m$ divides $a^n - 1$, \label{ass:order-1-mod-4:2m} 
    \item $\nu(a^n - 1)$ is not less than $m$, \label{ass:order-1-mod-4:nu-an} 
    \item $\nu(n)$ is not less than $m - \nu(a - 1)$, and \label{ass:order-1-mod-4:nu-n} 
    \item $k$ divides~$n$. \label{ass:order-1-mod-4:k} 
  \end{enumerate} 
  First,
  assertions~\ref{ass:order-1-mod-4:2m} and \ref{ass:order-1-mod-4:nu-an} are clearly equivalent.
  Second,
  Theorem~\ref{thm:val-adic-2-alt} yields
  $$
  \nu(a^n  - 1) = \nu(n) + \nu(a - 1)
  $$
  because
  $$
  a^n  - 1 = S(n, a) (a - 1) \, . 
  $$
  Therefore,
  assertions~\ref{ass:order-1-mod-4:nu-an} and \ref{ass:order-1-mod-4:nu-n} are equivalent.
  Third and last,
  let us prove that
  assertions~\ref{ass:order-1-mod-4:nu-n} and \ref{ass:order-1-mod-4:k} are equivalent. 
  If $a \equiv 1 \pmod{2^m}$
  then both assertions~\ref{ass:order-1-mod-4:nu-n} and \ref{ass:order-1-mod-4:k} hold true
  because $m - \nu(a - 1) \le 0$ and $k = 1$.
  Let us now assume $a \not \equiv 1 \pmod{2^m}$.
  Then, we have
  $\nu(a - 1) < m$,
  $d = 2^{\nu(a - 1)}$, 
  $k = 2^{m - \nu(a - 1)}$, and
  $\nu(k) = m - \nu(a - 1)$.
  Therefore, assertions~\ref{ass:order-1-mod-4:nu-n} and \ref{ass:order-1-mod-4:k} are equivalent.
\end{proof}




% \begin{theorem}
%   Let $b \in \bZ$ and let $m \in \bN$.
%   \begin{enumerate}
%   \item If $m \ge 2$ then the multiplicative order of $4 b + 1$  modulo $2^m$ is equal to
%     $$
%     \frac{2^{m - 2}}{\gcd(b, 2^{m - 2})} \,. 
%     $$
%   \item If $m \ge 3$ then the multiplicative order of $4b - 1$ modulo $2^m$ is equal to
%     $$
%     \frac{2^{m - 2}}{\gcd(b, 2^{m - 3})} \,. 
%     $$
%   \end{enumerate}
% \end{theorem}


\begin{theorem} \label{thm:order-trois-mod-4}
  Let $a \in \bZ$ and let $m \in \bN$ be such that be such that $a$ is odd and $a \not\equiv -1 \pmod{2^m}$.
  The multiplicative order of $a$ modulo $2^m$ is equal to that of $- a$.
\end{theorem}





\begin{proof}
  Put $d = \gcd(a - 1, 2^{m - 1})$
  Let us first assume $a \equiv - 1 \pmod{2^m}$.  
  Then, $\gcd(a + 1, 2^{m - 1}) $

   for each $x \in \bZ$.
  Since $4$ divides $- a - 1 = - (a + 1)$,
  Theorem~\ref{thm:order-un-mod-4} ensures that the multiplicative order of $- a$ modulo $2^m$ is equal to
  $2^m \mathbin{/} d(- a)$.
  Since $\gcd(x, y) = \gcd(- x, y)$ for every $x$, $y \in \bZ$, we have $d(- a) = \gcd(a + 1, 2^m)$.
  It remains to prove that the multiplicative order of
  Since  $d(- a) = 2^m$ 
\end{proof}

$$
\frac{b^n - a^n}{b - a} =  \sum_{k = 0}^{n - 1}a^k b^{n - 1 - k} 
$$
$$
\frac{{(a + p)}^{p - 1} - a^{p - 1}}{p}  =  \sum_{k = 0}^{p - 2} {(a +  p)}^k a^{p - 2 - k}
$$

$$
\frac{a^{p - 1} - 1}{p} = \frac{a - 1}{p} \sum_{k = 0}^{p - 2} a^k 
$$

$$
(p - 2) a^{p - 2} = 
$$

$$
a^{p - 1}  = 1 + k p^2 
$$

\begin{theorem}
  Let $a \in \bZ$ and let $n$, $p \in \bN$.
  If $p$ is prime and if $p$ divides $S(p, a)$ then $p$ divides $a - 1$.
  \end{theorem}

  \begin{proof}
  Assume that $p$ is prime.
  Straightforward computations yield
  $$(a - 1) (S(p, a) - 1) = a^p - a$$
  and
  Fermat's little theorem ensures that $p$ divides $a^p - a$.
  Therefore, $p$ divides $a - 1$ or $p$ divides $S(p, a) - 1$.
  Besides, the latter cannot happen if $p$ divides $S(p, a)$.
\end{proof}


\begin{theorem}
  Let $a \in \bZ$ and let $n \in \bNast$.
  \begin{enumerate}
  \item If $a$ is even or if $n$ is odd then $S(n, a)$ is odd.
  \item If $a$ is odd and if $n$ is even then $\nu(S(n, a)) = \nu(n) + \nu(a + 1) - 1$.
  \end{enumerate}
\end{theorem}

\begin{proof}
  First, assume that $a$ is even.
  Then, $a^n - 1$ is odd.
  Besides, $S(n, a)$ divides $a^n - 1$.
  Therefore, $S(n, a)$ is odd.

  Second, assume that both $a$ and $n$ are odd.
  Then, every power of $a$ is odd.
  Besides, $S(n, a)$ can be written as the sum of $n$ powers of~$a$.
  Therefore, $S(n, a)$ is odd as the sum of an even number of odd numbers.


  Third and last, assume that $a$ is odd and that $n$ is even.
  Then, $4$ divides $(a + 1)(a - 1) = a^2 - 1$,
  and thus Theorem~\ref{thm:val-adic} ensures
  $$
  \nu(S( n \mathbin{/} 2, a^2)) = \nu ( n  \mathbin{/} 2 ) = \nu(n)  - 1  \, .
  $$
  Besides, $S(n, a)$ can be written as 
  $$
  S(n, a) = S( n  \mathbin{/} 2, a^2) (a + 1) \, . 
  $$
  It follows
  $$
  \nu(S(n, a)) =  \nu(S( n \mathbin{/} 2, a^2)) + \nu(a + 1) =\nu(n) + \nu(a + 1) - 1 \, .
  $$
\end{proof}



\begin{theorem}
  For every $m \in \bNast$ and every $a \in \bZ$, 
  $m$ divides $S(m, a)$  if, and only if, $m$ divides $a^m - 1$.
\end{theorem}

\begin{proof}
  The ``only if part'' is trivial.
  Assume that  $m$ divides $S(m, a)$
\end{proof}


Assume that  $m$ and $\phi(m)$ are coprime.
Let $a \in \bZ$ be such that $m$ divides $S(m, a)$.
In particular, $m$ divides $a^m - 1$.
There exist $u$, $v \in \bZ$ such that 
$$
u m + v \phi(m) = 1
$$

$$
a^m 
$$
$m$ divides $a^m - 1$

Assume that $m$ and $\phi(m)$ are not coprime.
Then, there exists a prime number $p$ such that $p$ divides $\gcd(m, \phi(m))$,
and subsequently,
Cauchy's theorem ensures that there exists $a \in \bZ$ such that the multiplicative order of $a$ modulo $m$ equals~$p$.
Let $n \in \bN$ be such that $m = np$.
$$
(a - 1) S(m, a) = a^m - 1
$$

$p$ divides $a - 1$

Assume there exists $a \in \bZ$ such that $m$ does not divide $a - 1$ and $m$ divides $S(m, a)$.
Then there exists a prime number $p$ such that $p$ divides $m$ and $p$ does not divide $a - 1$


\begin{exercise}
   Assume that $p$ is not prime.
    Prove that $a$ can be chosen in such a way that $p$ divides $S(p, a)$
    Does every prime factor of $p$ divides $a - 1$?
\end{exercise} 


\begin{solution}
  Therefore, the first part of the exercise is complete.
  Let us now deal with the second part.
  Assume 
  Let $a \in \bZ$ and let $p \in \bN$ be such that
  $p$ is even, $p$ is not a power of $2$,  and $p$ divides $a + 1$:
  Let $a \in \bZ$ be such that  $a \equiv - 1 \pmod{p}$.
  Let $n \in \bN$.
  % Since $a \equiv -1 \pmod{p}$,
  % we have
  % $$
  % S(n + 1, - 1) \equiv 1 - S(n, - 1) \pmod{p}
  % $$
  % for every $n \in \bN$,
  % and thus
  
  % Since $p$ is even, $S(p, -1) =  \frac{1}{2}{(- 1)}^p - 1}$
  Since 
  $$
  S(n, -1) = \frac{{(- 1)}^n - 1}{2}
  $$
  and since 
  $$S(n, - 1) \equiv S(n, a) \pmod{p} \,, $$
  we have
  $$
  S(n + 2, a) \equiv S(n) \pmod{p} 
  $$
  However, at least one prime factor of $p$ does not divide $a - 1$ because  
  $p$ is not a power of $2  = \gcd(p, a - 1)$.
\end{solution}


$$
(a - 1) S(n, a) = (1 - a^n)
$$




\begin{theorem}
  Let $a \in \bZ$ and let $n \in \bNast$.
  If $4$ divides $a - 1$ then the multiplicative order of $a$ modulo $2^n$ is equal to 
 
\end{theorem}
